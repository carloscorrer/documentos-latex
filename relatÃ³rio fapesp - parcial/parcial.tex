\documentclass[12pt]{article}

%imagens
\usepackage{tikz,graphicx}
\usepackage{caption,subcaption}
\usepackage{float}

%matemáticas
\usepackage{amsmath,amsfonts,amssymb,amsthm}
\usepackage[ISO]{diffcoeff}
\usepackage{braket,esint}

%formatação
\usepackage[
 a4paper,
 tmargin=3.3cm,
 bmargin=4.2cm,
 lmargin=3cm,
 rmargin=3cm,
]{geometry}
\usepackage[T1]{fontenc}
\usepackage{inputenc}
\usepackage[portuguese]{babel}
\usepackage{lmodern}
\usepackage{indentfirst}
\usepackage{orcidlink}
\usepackage{csquotes}
\usepackage{bbold}

%referências
\usepackage{hyperref}
\usepackage[capitalize, brazilian, nameinlink]{cleveref}
\hypersetup{ 
colorlinks = true,
linkcolor = blue,
filecolor = black,
urlcolor = black
}

%funções
\newcommand{\m}{\mathcal{M}}
\newcommand{\s}{\mathcal{S}}
\newcommand{\kg}{\left(\nabla^a\nabla_a+m^2\right)}
\newcommand{\dd}{\mathrm{d}}
\newcommand{\innerkg}[2]{\langle#1,#2\rangle_{\text{KG}}}
\newcommand{\h}{\mathcal{H}}

\title{Radiação, Aceleração e o Efeito Unruh}
\author{Carlos H. Correr da Silva\orcidlink{0000-0002-3598-4412}}


\begin{document}

\begin{titlepage}
    \begin{center}
        \vspace{5cm}
        \makeatletter
        {\huge\@title}\\
        \vspace{0.5cm}
        {\LARGE \href{https://bv.fapesp.br/pt/pesquisa/buscador/?q2=(id_pesquisador_exact%3A724334)%20AND%20(bolsa_exact:%22Bolsas%20no%20Brasil%22%20AND%20situacao:%22Em%20andamento%22)}{Processo: 24/01141-1}}\\
        \hrulefill \\
        \vspace{0.5cm}
        {\LARGE\@author
        \vspace{0.5cm}\\
        Orientador: André G. S. Landulfo}\\
        \vspace{2cm}
        {\LARGE Vigência: 01/05/2024 a 30/04/2025\\
        \vspace{0.2cm}
        Período coberto: 01/05/2024 a 10/10/2024}
    \end{center}
    \vspace{6cm}
    Resumo: A conexão entre aceleração e radiação e sua relação com o princípio da equivalência vêm intrigando a comunidade científica a décadas. Mais recentemente, esta questão foi investigada da perspectiva da mecânica quântica no âmbito da teoria quântica de campos em espaços-tempos curvos e foi encontrado o que parece ser uma conexão impressionante entre o  bremsstrahlung (um efeito clássico) e o efeito Unruh (um efeito puramente quântico). Mais do que isso, os chamados fótons de Rindler de {\em energia nula} desempenham um papel surpreendente, porém crucial, em tal conexão. Neste projeto de iniciação científica, pretendemos estudar os aspectos clássicos e quânticos da radiação emitida por uma carga (escalar) acelerada e o papel exato desempenhado pelos fótons de Rindler de energia nula (já no contexto clássico). Tal análise servirá como "cavalo de batalha" para introduzir o estudante à teoria quântica de campos em espaços-tempos curvos e seus mais diversos aspectos conceituais.
\end{titlepage}

\hrule
\tableofcontents
\hspace*{2mm}
\hrule

\section{Graduação}
Ao longo do período coberto, conclui o meu 5° semestre da graduação, em que cursei as seguintes  matérias: Mecânica Quântica I, Eletromagnetismo I, Termodinâmica, Mecânica I e Física Experimental V. As bases teóricas de quântica estudadas para o projeto complementaram-se e solidificaram-se com o andamento da disciplina cursada. Além disso, a matéria de eletromagnetismo forneceu o arcabouço para o estudo da radiação emitida por uma carga acelerada. 

No fim do período, estou na metade do 6° semestre, cursando as disciplinas: Relatividade Geral, Eletromagnetismo II, Grupos e Tensores, Mecânica Estatística e Física Computacional I. A matéria de relatividade está reforçando os conceitos que estudei antes do início do projeto e complementando o incentivo com diversos exercícios. Enquanto isso, eletromagnetismo está enfatizando a importância dos potenciais eletromagnéticos e as liberdades de escolha que permitem obter equações de onda para as grandezas relevantes, mantendo a física do sistema de estudo.

Ainda por cima, participei dos Journal Clubs do grupo, que permite o contato com diversas tópicos atuais de pesquisa na área. Também apresentei um pôster sobre o Efeito Unruh na VI Escola Jayme Tiomno, uma escola de inverno do IFUSP, na qual sou um dos organizadores.

\section{Teoria quântica de um campo escalar em espaços-curvos}
\subsection*{Campo de Klein-Gordon}
Seja \((\m,g_{ab})\) um espaço-tempo globalmente hiperbólico, então, podemos decompor \(\m\) em uma folição parametrizada por uma função diferenciável \(t:\m\to\mathbb{R}\) de superfícies de Cauchy \(\Sigma_t\) tal que \(\m\cong\mathbb{R}\times\Sigma_t\).

Um campo escalar (spin-0) e massa \(m\) nesse espaço-tempo satisfaz a equação de Klein-Gordon,
\begin{equation}
    \kg\phi=0,
    \label{eq:kg}
\end{equation}
em que \(\nabla^a\) é a derivada covariante compatível com a métrica. O momento conjugado ao campo é dado por
\begin{equation}
    \pi=n^a\nabla_a\phi.
\end{equation}
Assim, dada a hierbolicidade global do campo, uma solução da equação de Klein-Gordon está unicamente definido dadas condições iniciais, isto é, funções suaves numa superfície de Cauchy para o valor do campo e seu momento.
Agora, dadas duas soluções \(\phi_1\) e \(\phi_2\) da \cref{eq:kg}, definimos a forma simplética
\begin{equation}
    \Omega(\phi_1,\phi_2)\equiv\int_{\Sigma_t}\dd^3x\sqrt{h}\left(\phi_2n^a\nabla_a\phi_1-\phi_1n^a\nabla_a\phi_2\right),
\end{equation}
em que \(h_{ab}=g_{ab}\vert_{\Sigma_t}\). Notando que duas superfícies de Cauchy definem um volume no espaço-tempo, podemos usar a lei da divergência de Gauss para mostrar que a forma independe do parâmetro \(t\) da superfície, ou seja,
\begin{equation}
    \diff{}{t}\Omega(\phi_1,\phi_2)=0.
\end{equation}

Finalmente, se \(\s\) é o espaço de soluções da \cref{eq:kg}, o par \((\s,\Omega)\) é suficiente para realizar a quantização do campo.

\subsection*{Quantização}
Primeiro, complexificamos o espaço de soluções, isto é, \(\s\to\s^{\mathbb{C}}\) e definimos nele o produto interno de Klein-Gordon,
\begin{equation}
    \innerkg{f_1}{f_2}\equiv-i\Omega(f_1^*,f_2),\; f_1,f_2\in\s^{\mathbb{C}}.
\end{equation}
Agora, basta tomar um subconjunto \(\h\subset\s^{\mathbb{C}}\) que satisfaça as seguintes propriedades:
\begin{enumerate}
    \item \(\s^{\mathbb{C}}=\h\oplus\overline{\h}\), em que \(\overline{\h}\) é o espaço conjugado a \(\h\)
    \item \(\innerkg{\cdot}{\cdot}\) é positivo definido em \(\h\) e, portanto, \(\left(\h,\innerkg{\cdot}{\cdot}\right)\) é um espaço de Hilbert.
    \item Para todo \(f_1\in\h\) e \(f_2\in\overline{\h}\) temos \(\innerkg{f_1}{f_2}=0\).
\end{enumerate}

Nessa construção, \(\h\) é o espaço de \enquote{1-partícula}\footnote{A seguir veremos que o conceito de partícula é delicado no contexto de TQCEC}, portanto, o espaço de estados será dado pelo espaço de Fock \(\mathcal{F}(\h)\) associado à escolha de \(\h\).

É possível definir naturalmente operadores de aniquilação e de criação, \(a(\xi^*)\) e \(a^\dagger(\xi)\) no espaço de Fock simétrico, em que \(\xi\in\h\) representa o modo que está sendo criado ou aniquilado. Além disso, diramente da definição, obtemos as relações de comutação
\begin{subequations}
    \begin{align}
        &\relax\left[a(\xi^*),a^{\dagger}(\eta)\right]=\innerkg{\xi}{\eta}\mathbb{1},\\
        &\relax\left[a(\xi^*),a(\eta^*)\right]=0,\\
        &\relax\left[a^{\dagger}(\xi),a^{\dagger}(\eta)\right]=0,
    \end{align}
\end{subequations}
para todo \(\xi,\eta\in\h\). Além disso, a construção leva a uma noção de vácuo associado com o estado que é anulado por qualquer operador de aniquilação,
\begin{equation}
    a(\xi^*)\ket{0}=0,\;\forall\xi\in\h,
\end{equation}
do qual podemos definir um estado de \(n\) partículas com modo \(\xi\) como
\begin{equation}
    \ket{n_\xi}\equiv\frac{\left[a^{\dagger}(\xi)\right]^n}{\sqrt{n!}}\ket{0}.
\end{equation}

Finalmente, podemos expandir o campo (quantizado) num conjunto completo de soluções \(\{u_j\}\) de frequência positiva como
\begin{equation}
    \hat{\phi}(x)=\sum_j\left[u_ja(u_j ^*)+u_j^*a^{\dagger}(u_j)\right].
\end{equation}

É conveniente definir os projetores \(K:\s^{\mathbb{C}}\to\h\) e \(\overline{K}:\s^\mathbb{C}\to\overline{\h}\) que dado uma solução \(\varphi=\varphi^++\varphi^-\) com \(\varphi^+\in\h\) e \(\varphi^-\in\overline{\h}\), a ação desses operadores é definida como
\begin{equation}
    K\varphi=\varphi^+\hspace{1cm}\text{e}\hspace{1cm}\overline{K}\varphi=\varphi^-.
\end{equation}

Portanto, expandindo \(\varphi\) no base de soluções \(\{u_j\}\), obtemos a ação explícita dos projetores,
\begin{equation}
    \varphi=\sum_j\left[\innerkg{u_j}{\varphi}u_j-\innerkg{u_j^*}{\varphi}u_j^*\right],
\end{equation}
então,
\begin{equation}
    K\varphi=\innerkg{u_j}{\varphi}u_j\hspace{1cm}\text{e}\hspace{1cm}\overline{K}\varphi=-\sum_j\innerkg{u_j^*}{\varphi}u_j^*.
\end{equation}

\subsection*{Operador campo}
Seja \(f\in C_0^{\infty}(\m)\), ou seja, uma função suave de suporte compacto. Denotamos por \(G_A(x,x')\) e \(G_R(x,x')\) as funções de Green avançada e retardada do operador de Klein-Gordon. Assim, temos as soluções retardada e avançada com fonte \(f\) dadas por
\begin{subequations}
    \begin{align}
        &Rf(x)\equiv\int_\m\dd^4x'\sqrt{-g}G_R(x,x')f(x')\\
        &Af(x)\equiv\int_\m\dd^4x'\sqrt{-g}G_A(x,x')f(x'),
    \end{align}
\end{subequations}
em que \(Rf\) tem suporte no futuro do suporte de \(f\) e \(Af\) no passado. Note que podemos definir uma solução da \cref{eq:kg} a partir das soluções retardada e avançada como
\begin{equation}
    Ef(x)\equiv Af(x)-Rf(x).
\end{equation}
De fato esta é uma solução homogênea, dado que
\begin{subequations}
    \begin{align}
        \kg Ef(x)&=\kg Af(x)-\kg Rf(x)\\
        &=f(x)-f(x)=0.
    \end{align}
\end{subequations}

Essa solução é extramemente útil pois goza da propriedade que dada um solução qualquer \(\varphi\) da \cref{eq:kg}, o produto interno desta com \(Ef\) pode ser escrito como uma integral no espaço-tempo,
\begin{equation}
    -i\innerkg{\varphi}{Ej}=\int_\m\dd^4xf(x)\varphi^*(x).
    \label{eq:int-inner}
\end{equation}
Além disso, se identificarmos \(E:C_0^{\infty}(\m)\to\s\) como um mapa de funções em soluções, temos que esse mapa é sobrejetor, isto é, toda solução está associada a uma função suave da variedade.

Agora, definimos o operador campo associado a uma função \(f\) como
\begin{subequations}
    \begin{align}
        \hat{\phi}(f)\equiv\int_\m\dd^4x\sqrt{-g}\phi(x)f(x).
    \end{align}
\end{subequations}

Note que é possível escrevê-lo a partir dos operadores de aniquilação e criação expandindo o campo nos modos
\begin{subequations}
    \begin{align}
        \hat{\phi}(f)&=\sum_j\left[\int_\m\dd^4x\sqrt{-g}u_jf(x)a(u_j^*)+\int_\m\dd^4x\sqrt{-g}u_j^*f(x)a^{\dagger}(u_j)\right]\\
        &=\sum_j\left[-i\innerkg{u_j^*}{Ef}a(u_j^*)-i\innerkg{u_j}{Ef}a^\dagger(u_j)\right]\\
        &=ia\left(-i\sum_j\innerkg{u_j^*}{Ef}u^*_j\right)-ia^\dagger\left(\sum_j\innerkg{u_j}{Ef}u_j\right)\\
        &=ia\left(\overline{K}Ef\right)-a^\dagger\left(KEf\right),
    \end{align}
\end{subequations}
onde na segunda linha usamos a propriedade da \cref{eq:int-inner} para trocar as integrais espaciais por produtos internos e, a partir da linearidade dos operadores, identificamos a parte de frequência positiva e negativa do modo \(Ef\). Denotando \(\overline{K}Ef\) por \(KEf^*\) temos,
\begin{equation}
    \hat{\phi}(f)=ia\left(KEf^*\right)-ia^\dagger\left(KEf\right).
\end{equation}

\subsection*{Relações de comutação}
Seja \(f,g\in C_0^{\infty}(\m)\), então
\begin{subequations}
    \begin{align}
        \relax\left[\hat{\phi}(f),\hat{\phi}(g)\right]&=\left[ia\left(KEf^*\right)-ia^\dagger\left(KEf\right),ia\left(KEg^*\right)-ia^\dagger\left(KEg\right)\right]\\
        &=\left[a\left(KEf^*\right),a^\dagger\left(KEg\right)\right]-\left[a\left(KEg^*\right),a^\dagger\left(KEf\right)\right]\\
        &=\innerkg{KEf}{KEg}-\innerkg{KEg}{KEf}\\
        &=\innerkg{KEf}{KEg}-\innerkg{KEf}{KEg}^*.
    \end{align}
    \label{eq:comuta}
\end{subequations}

Agora, note que podemos usar da forma simplética do espaço para escrever
\begin{subequations}
    \begin{align}
        \innerkg{KEf^*}{KEg^*}&=-i\Omega(KEf,KEg^*)\\
        &=i\Omega(KEg^*,KEf)\\
        &=-\innerkg{KEg}{KEf}\\
        &=-\innerkg{KEf}{KEg}^*.
    \end{align}
\end{subequations}

Portanto, definimos
\begin{subequations}
    \begin{align}
        -iE(f,g)&\equiv\innerkg{Ef}{Eg}\\
        &=\innerkg{KEf+KEf^*}{KEg+KEg^*}\\
        &=\innerkg{KEf}{KEg}+\innerkg{KEf^*,KEg^*}\\
        &=\innerkg{KEf}{KEg}-\innerkg{KEf}{KEg}^*,
        \label{eq:def-E}
    \end{align}
\end{subequations}
de modo que, comparando com a \cref{eq:comuta}, temos as relações de comutação para os campos dadas por
\begin{equation}
    \relax\left[\hat{\phi}(f),\hat{\phi}(g)\right]=-E(f,g).
\end{equation}


\section{Efeito Unruh}
\subsection*{Espaços-tempos estacionários}
Uma das consequências paradigmáticas da teoria quântica de campos em espaços-tempos curvos é que arbitrariedade da escolha do espaço \(\h\) de 1-partícula, leva a conclusão que o conceito de partícula não é covariante. Na realidade, este conceito só faz sentido quando associado a um campo de Killing, isto é, a alguma simetria do espaço em questão.

Dado um espaço-tempo \((\m,g_{ab})\) globalmente hiperbólico e estacionário, isto é, que admite um campo de Killing tipo-tempo \(\xi\) com grupo de órbitas \(\phi_\xi^t:\mathbb{R}\to\m\), temos uma escolha natural de quantização. O espaço de Hilbert \(\h\) \enquote*{preferencial} é aquele cujas soluções são de frequência positiva com respeito ao tempo de Killing \(t\), uma função que satisfaz
\begin{equation}
    \xi^a\nabla_at=1.
\end{equation}
A saber, estudando a resposta de um detector de dois-níveis seguindo as órbitas do campo de Killing, chegamos a conclusão de que suas excitações e desexcitações estão associadas com a absroção e emissão da definição natural de partícula. Esse resultado é motivador para a interpretação dada nesse tipo de espaço-tempo.
\subsection*{Campos de Killing no espaço de Minkowski}
O alto grau de simetria do espaço-tempo de Minkowski, descrito pela métrica,
\begin{equation}
    \dd s^2=-\dd t^2+\dd x^2+\dd y^2+\dd z^2,
\end{equation}
carrega consigo diversos campos de Killing. Considere os seguintes,
\begin{equation}
    \xi^a=\left(\partial_t\right)^a\hspace{1cm}\text{e}\hspace{1cm}\chi^a=a\left[x\left(\partial_t\right)^a+t\left(\partial_x\right)^a\right].
\end{equation}
O primeiro, tem suas órbitas associadas com translação temporal e imediatamente segue que um observador com 4-velocidade dada por \(\xi^a\) tem 4-aceleração nula, logo correspondem a observadores inerciais. Agora, se um observador tem 4-velocidade dada por \(\chi^a\) normalizado, ou seja, \(u^a=\chi^a/B^2\) onde \(B^2=-\chi^a\chi_a=a^2(x^2-t^2)\), sua 4-aceleração é
\begin{subequations}
    \begin{align}
        A^a&=\frac{1}{B^2}\chi^b\nabla_b\chi^a=-\frac{1}{B^2}\chi_b\nabla^a\chi^b=-\frac{1}{2B^2}\nabla^a\left(\chi^b\chi_b\right)\\
        &=\frac{1}{2B^2}\nabla^aB^2=\nabla^a\ln{B}.
    \end{align}
\end{subequations}
Portanto, a sua magnitude é
\begin{equation}
    A=\left(A^aA_a\right)^{\frac{1}{2}}=(x^2-t^2)^{-\frac{1}{2}}=\frac{a}{B},
\end{equation}
ou seja, esses observadores estão uniformemente acelerados. Além disso, note que o campo \(\chi^a\) divide o espaço-tempo em 4 regiões delimitadas pelas superfícies em que ele se anula, \(\mathfrak{h}_A\) e \(\mathfrak{h}_B\):
\begin{equation}
    \begin{aligned}
        &\text{Região I}=I^-\left(\mathfrak{h}_A\right)\cap I^+\left(\mathfrak{h}_B\right)\hspace{2cm}\text{Região II}=I^+\left(\mathfrak{h}_A\right)\cap I^-\left(\mathfrak{h}_B\right)\\
        &\text{Região III}=J^-(S)\hspace{3.75cm}\text{Região IV}=J^+(S)
    \end{aligned}
\end{equation}

Na
\section{Radiação por uma carga acelerada}

\end{document}