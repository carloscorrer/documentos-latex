\documentclass[12pt]{article}

%imagens
\usepackage{tikz,graphicx}
\usepackage{caption,subcaption}
\usepackage{float}

%matemáticas
\usepackage{amsmath,amsfonts,amssymb,amsthm}
\usepackage[ISO]{diffcoeff}
\usepackage{braket,esint,siunitx}

%formatação
\usepackage[
 a4paper,
 tmargin=3.3cm,
 bmargin=4.2cm,
 lmargin=3cm,
 rmargin=3cm,
]{geometry}
\usepackage[T1]{fontenc}
\usepackage{inputenc}
\usepackage[portuguese]{babel}
\usepackage{lmodern}
\usepackage{indentfirst}
\usepackage{orcidlink}
\usepackage{csquotes}
\usepackage{bbold}

%referências
\usepackage{hyperref}
\usepackage[capitalize, brazilian, nameinlink]{cleveref}
\hypersetup{ 
colorlinks = true,
linkcolor = blue,
filecolor = black,
urlcolor = black
}

%funções
\newcommand{\m}{\mathcal{M}}
\newcommand{\s}{\mathcal{S}}
\newcommand{\kg}{\left(\nabla^a\nabla_a+m^2\right)}
\newcommand{\dd}{\mathrm{d}}
\newcommand{\innerkg}[2]{\langle#1,#2\rangle_{\text{KG}}}
\newcommand{\h}{\mathcal{H}}
\newcommand{\x}{\mathbf{x}}
\newcommand{\xq}{\mathbf{x}_q}
\newcommand{\tr}{t_{\text{ret}}}
\newcommand{\n}{\mathbf{n}}
\newcommand{\er}{\hat{\mathbf{e}}_r}
\newcommand{\Ttot}{T_{\text{tot}}}
\newcommand{\xperp}{\mathbf{x}_{\perp}}
\newcommand{\supp}[1]{\text{supp}\;#1}
\newcommand{\wu}[1]{w^1_{#1\mathbf{k}_\perp}}
\newcommand{\wv}[1]{w^2_{#1\mathbf{k}_\perp}}
\newcommand{\wsig}[1]{w^{\sigma}_{#1\mathbf{k}_\perp}}

\title{Radiação, Aceleração e o Efeito Unruh}
\author{Carlos H. Correr da Silva\orcidlink{0000-0002-3598-4412}}


\begin{document}

\begin{titlepage}
    \begin{center}
        \vspace{5cm}
        \makeatletter
        {\huge\@title}\\
        \vspace{0.5cm}
        {\LARGE \href{https://bv.fapesp.br/pt/pesquisa/buscador/?q2=(id_pesquisador_exact%3A724334)%20AND%20(bolsa_exact:%22Bolsas%20no%20Brasil%22%20AND%20situacao:%22Em%20andamento%22)}{Processo: 24/01141-1}}\\
        \hrulefill \\
        \vspace{0.5cm}
        {\LARGE\@author
        \vspace{0.5cm}\\
        Orientador: André G. S. Landulfo}\\
        \vspace{2cm}
        {\LARGE Vigência: 01/05/2024 a 30/04/2025\\
        \vspace{0.2cm}
        Período coberto: 01/05/2024 a 10/10/2024}
    \end{center}
    \vspace{6cm}
    Resumo: A conexão entre aceleração e radiação e sua relação com o princípio da equivalência vêm intrigando a comunidade científica a décadas. Mais recentemente, esta questão foi investigada da perspectiva da mecânica quântica no âmbito da teoria quântica de campos em espaços-tempos curvos e foi encontrado o que parece ser uma conexão impressionante entre o  bremsstrahlung (um efeito clássico) e o efeito Unruh (um efeito puramente quântico). Mais do que isso, os chamados fótons de Rindler de {\em energia nula} desempenham um papel surpreendente, porém crucial, em tal conexão. Neste projeto de iniciação científica, pretendemos estudar os aspectos clássicos e quânticos da radiação emitida por uma carga (escalar) acelerada e o papel exato desempenhado pelos fótons de Rindler de energia nula (já no contexto clássico). Tal análise servirá como "cavalo de batalha" para introduzir o estudante à teoria quântica de campos em espaços-tempos curvos e seus mais diversos aspectos conceituais.
\end{titlepage}

\hrule
\tableofcontents
\hspace*{2mm}
\hrule

\section{Graduação}
Ao longo do período coberto, conclui o meu 5° semestre da graduação, em que cursei as seguintes  matérias: Mecânica Quântica I, Eletromagnetismo I, Termodinâmica, Mecânica I e Física Experimental V. As bases teóricas de quântica estudadas para o projeto complementaram-se e solidificaram-se com o andamento da disciplina cursada. Além disso, a matéria de eletromagnetismo forneceu o arcabouço para o estudo da radiação emitida por uma carga acelerada. 

No fim do período, estou na metade do 6° semestre, cursando as disciplinas: Relatividade Geral, Eletromagnetismo II, Grupos e Tensores, Mecânica Estatística e Física Computacional I. A matéria de relatividade está reforçando os conceitos que estudei antes do início do projeto e complementando o incentivo com diversos exercícios. Enquanto isso, eletromagnetismo está enfatizando a importância dos potenciais eletromagnéticos e as liberdades de escolha que permitem obter equações de onda para as grandezas relevantes, mantendo a física do sistema de estudo.

Ainda por cima, participei dos Journal Clubs do grupo, que permite o contato com diversas tópicos atuais de pesquisa na área. Também apresentei um pôster sobre o Efeito Unruh na VI Escola Jayme Tiomno, uma escola de inverno do IFUSP, na qual sou um dos organizadores.

\section{Teoria quântica de um campo escalar em espaços-curvos}
\subsection*{Campo de Klein-Gordon}
Seja \((\m,g_{ab})\) um espaço-tempo globalmente hiperbólico, então, podemos decompor \(\m\) em uma folição parametrizada por uma função diferenciável \(t:\m\to\mathbb{R}\) de superfícies de Cauchy \(\Sigma_t\) tal que \(\m\cong\mathbb{R}\times\Sigma_t\).

Um campo escalar (spin-0) e massa \(m\) nesse espaço-tempo satisfaz a equação de Klein-Gordon,
\begin{equation}
    \kg\phi=0,
    \label{eq:kg}
\end{equation}
em que \(\nabla^a\) é a derivada covariante compatível com a métrica. O momento conjugado ao campo é dado por
\begin{equation}
    \pi=n^a\nabla_a\phi,
\end{equation}
e, consequentemente, tensor de energia-momento
\begin{equation}
    T_{ab}=\nabla_a\phi\nabla_b\phi-\frac{1}{2}g_{ab}\nabla^c\phi\nabla_c\phi.
\end{equation}
Assim, dada a hiperbolicidade global do campo, uma solução da equação de Klein-Gordon está unicamente definido dadas condições iniciais, isto é, funções suaves numa superfície de Cauchy para o valor do campo e seu momento.
Agora, dadas duas soluções \(\phi_1\) e \(\phi_2\) da \cref{eq:kg}, definimos a forma simplética
\begin{equation}
    \Omega(\phi_1,\phi_2)\equiv\int_{\Sigma_t}\dd^3x\sqrt{h}\left(\phi_2n^a\nabla_a\phi_1-\phi_1n^a\nabla_a\phi_2\right),
\end{equation}
em que \(h_{ab}=g_{ab}\vert_{\Sigma_t}\). Notando que duas superfícies de Cauchy definem um volume no espaço-tempo, podemos usar a lei da divergência de Gauss para mostrar que a forma independe do parâmetro \(t\) da superfície, ou seja,
\begin{equation}
    \diff{}{t}\Omega(\phi_1,\phi_2)=0.
\end{equation}

Finalmente, se \(\s\) é o espaço de soluções da \cref{eq:kg}, o par \((\s,\Omega)\) é suficiente para realizar a quantização do campo.

\subsection*{Quantização}
Primeiro, complexificamos o espaço de soluções, isto é, \(\s\to\s^{\mathbb{C}}\) e definimos nele o produto interno de Klein-Gordon,
\begin{equation}
    \innerkg{f_1}{f_2}\equiv-i\Omega(f_1^*,f_2),\; f_1,f_2\in\s^{\mathbb{C}}.
\end{equation}
Agora, basta tomar um subconjunto \(\h\subset\s^{\mathbb{C}}\) que satisfaça as seguintes propriedades:
\begin{enumerate}
    \item \(\s^{\mathbb{C}}=\h\oplus\overline{\h}\), em que \(\overline{\h}\) é o espaço conjugado a \(\h\)
    \item \(\innerkg{\cdot}{\cdot}\) é positivo definido em \(\h\) e, portanto, \(\left(\h,\innerkg{\cdot}{\cdot}\right)\) é um espaço de Hilbert.
    \item Para todo \(f_1\in\h\) e \(f_2\in\overline{\h}\) temos \(\innerkg{f_1}{f_2}=0\).
\end{enumerate}

Nessa construção, \(\h\) é o espaço de \enquote{1-partícula}\footnote{A seguir veremos que o conceito de partícula é delicado no contexto de TQCEC}, portanto, o espaço de estados será dado pelo espaço de Fock simétrico \(\mathcal{F}_s(\h)\) associado à escolha de \(\h\).

É possível definir naturalmente operadores de aniquilação e de criação, \(a(\xi^*)\) e \(a^\dagger(\xi)\) no espaço de Fock, em que \(\xi\in\h\) representa o modo que está sendo criado ou aniquilado. Além disso, diretamente da definição, obtemos as relações de comutação
\begin{subequations}
    \begin{align}
        &\relax\left[a(\xi^*),a^{\dagger}(\eta)\right]=\innerkg{\xi}{\eta}\mathbb{1},\\
        &\relax\left[a(\xi^*),a(\eta^*)\right]=0,\\
        &\relax\left[a^{\dagger}(\xi),a^{\dagger}(\eta)\right]=0,
    \end{align}
\end{subequations}
para todo \(\xi,\eta\in\h\). Além disso, a construção leva a uma noção de vácuo associado com o estado que é anulado por qualquer operador de aniquilação,
\begin{equation}
    a(\xi^*)\ket{0}=0,\;\forall\xi\in\h,
\end{equation}
do qual podemos definir um estado de \(n\) partículas com modo \(\xi\) como
\begin{equation}
    \ket{n_\xi}\equiv\frac{\left[a^{\dagger}(\xi)\right]^n}{\sqrt{n!}}\ket{0}.
\end{equation}

Finalmente, podemos expandir o campo (quantizado) num conjunto completo de soluções \(\{u_j\}\) de frequência positiva como
\begin{equation}
    \hat{\phi}(x)=\sum_j\left[u_ja(u_j ^*)+u_j^*a^{\dagger}(u_j)\right].
\end{equation}

É conveniente definir os projetores \(K:\s^{\mathbb{C}}\to\h\) e \(\overline{K}:\s^\mathbb{C}\to\overline{\h}\) que dado uma solução \(\varphi=\varphi^++\varphi^-\) com \(\varphi^+\in\h\) e \(\varphi^-\in\overline{\h}\), a ação desses operadores é definida como
\begin{equation}
    K\varphi=\varphi^+\hspace{1cm}\text{e}\hspace{1cm}\overline{K}\varphi=\varphi^-.
\end{equation}

Portanto, expandindo \(\varphi\) no base de soluções \(\{u_j\}\), obtemos a ação explícita dos projetores,
\begin{equation}
    \varphi=\sum_j\left[\innerkg{u_j}{\varphi}u_j-\innerkg{u_j^*}{\varphi}u_j^*\right],
\end{equation}
então,
\begin{equation}
    K\varphi=\innerkg{u_j}{\varphi}u_j\hspace{1cm}\text{e}\hspace{1cm}\overline{K}\varphi=-\sum_j\innerkg{u_j^*}{\varphi}u_j^*.
\end{equation}

\subsection*{Operador campo}
Seja \(f\in C_0^{\infty}(\m)\), ou seja, uma função suave de suporte compacto. Denotamos por \(G_A(x,x')\) e \(G_R(x,x')\) as funções de Green avançada e retardada do operador de Klein-Gordon. Assim, temos as soluções retardada e avançada com fonte \(f\) dadas por
\begin{subequations}
    \begin{align}
        &Rf(x)\equiv\int_\m\dd^4x'\sqrt{-g}G_R(x,x')f(x')\\
        &Af(x)\equiv\int_\m\dd^4x'\sqrt{-g}G_A(x,x')f(x'),
    \end{align}
\end{subequations}
em que \(Rf\) tem suporte no futuro do suporte de \(f\) e \(Af\) no passado. Note que podemos definir uma solução da \cref{eq:kg} a partir das soluções retardada e avançada como
\begin{equation}
    Ef(x)\equiv Af(x)-Rf(x).
\end{equation}
De fato esta é uma solução homogênea, dado que
\begin{subequations}
    \begin{align}
        \kg Ef(x)&=\kg Af(x)-\kg Rf(x)\\
        &=f(x)-f(x)=0.
    \end{align}
\end{subequations}

Essa solução é extremamente útil pois goza da propriedade que dada um solução qualquer \(\varphi\) da \cref{eq:kg}, o produto interno desta com \(Ef\) pode ser escrito como uma integral no espaço-tempo,
\begin{equation}
    -i\innerkg{\varphi}{Ej}=\int_\m\dd^4xf(x)\varphi^*(x).
    \label{eq:int-inner}
\end{equation}
Além disso, se identificarmos \(E:C_0^{\infty}(\m)\to\s\) como um mapa de funções em soluções, temos que esse mapa é sobrejetor, isto é, toda solução está associada a uma função suave da variedade.

Agora, definimos o operador campo associado a uma função \(f\) como
\begin{subequations}
    \begin{align}
        \hat{\phi}(f)\equiv\int_\m\dd^4x\sqrt{-g}\phi(x)f(x).
    \end{align}
\end{subequations}

Note que é possível escrevê-lo a partir dos operadores de aniquilação e criação expandindo o campo nos modos
\begin{subequations}
    \begin{align}
        \hat{\phi}(f)&=\sum_j\left[\int_\m\dd^4x\sqrt{-g}u_jf(x)\hat{a}(u_j^*)+\int_\m\dd^4x\sqrt{-g}u_j^*f(x)\hat{a}^{\dagger}(u_j)\right]\\
        &=\sum_j\left[-i\innerkg{u_j^*}{Ef}\hat{a}(u_j^*)-i\innerkg{u_j}{Ef}\hat{a}^\dagger(u_j)\right]\\
        &=i\hat{a}\left(-i\sum_j\innerkg{u_j^*}{Ef}u^*_j\right)-i\hat{a}^\dagger\left(\sum_j\innerkg{u_j}{Ef}u_j\right)\\
        &=i\hat{a}\left(\overline{K}Ef\right)-i\hat{a}^\dagger\left(KEf\right),
    \end{align}
\end{subequations}
onde na segunda linha usamos a propriedade da \cref{eq:int-inner} para trocar as integrais espaciais por produtos internos e, a partir da linearidade dos operadores, identificamos a parte de frequência positiva e negativa do modo \(Ef\). Denotando \(\overline{K}Ef\) por \(KEf^*\) temos,
\begin{equation}
    \hat{\phi}(f)=i\hat{a}\left(KEf^*\right)-i\hat{a}^\dagger\left(KEf\right).
    \label{eq:campo}
\end{equation}

\subsection*{Relações de comutação}
Seja \(f,g\in C_0^{\infty}(\m)\), então
\begin{subequations}
    \begin{align}
        \relax\left[\hat{\phi}(f),\hat{\phi}(g)\right]&=\left[\hat{a}\left(KEf^*\right),\hat{a}^\dagger\left(KEg\right)\right]-\left[\hat{a}\left(KEg^*\right),\hat{a}^\dagger\left(KEf\right)\right]\\
        &=\innerkg{KEf}{KEg}-\innerkg{KEf}{KEg}^*.
    \end{align}
    \label{eq:comuta}
\end{subequations}

Agora, note que podemos usar da forma simplética do espaço para escrever
\begin{equation}
    \innerkg{KEf^*}{KEg^*}=-\innerkg{KEf}{KEg}^*.
\end{equation}

Portanto, usando que \(Ef=KEf+KEf^*\), definimos
\begin{subequations}
    \begin{align}
        -iE(f,g)&\equiv\innerkg{Ef}{Eg}\\
        &=\innerkg{KEf}{KEg}-\innerkg{KEf}{KEg^*},
        \label{eq:def-E}
    \end{align}
\end{subequations}
de modo que, comparando com a \cref{eq:comuta}, temos as relações de comutação para os campos dadas por
\begin{equation}
    \relax\left[\hat{\phi}(f),\hat{\phi}(g)\right]=-E(f,g).
\end{equation}


\section{Efeito Unruh}
\subsection*{Espaços-tempos estacionários}
Uma das consequências paradigmáticas da teoria quântica de campos em espaços-tempos curvos é que arbitrariedade da escolha do espaço \(\h\) de 1-partícula, leva a conclusão que o conceito de partícula não é covariante. Na realidade, este conceito só faz sentido quando associado a um campo de Killing, isto é, a alguma simetria do espaço em questão.

Dado um espaço-tempo \((\m,g_{ab})\) globalmente hiperbólico e estacionário, isto é, que admite um campo de Killing tipo-tempo \(\xi\) com grupo de órbitas \(\phi_\xi^t:\mathbb{R}\to\m\), temos uma escolha natural de quantização. O espaço de Hilbert \(\h\) \enquote*{preferencial} é aquele cujas soluções são de frequência positiva com respeito ao tempo de Killing \(t\), uma função que satisfaz
\begin{equation}
    \xi^a\nabla_at=1.
\end{equation}
A saber, estudando a resposta de um detector de dois-níveis seguindo as órbitas do campo de Killing, chegamos a conclusão de que suas excitações e desexcitações estão associadas com a absroção e emissão da definição natural de partícula. Esse resultado é motivador para a interpretação dada nesse tipo de espaço-tempo.
\subsection*{Campos de Killing no espaço de Minkowski}
O alto grau de simetria do espaço-tempo de Minkowski, descrito pela métrica,
\begin{equation}
    \dd s^2=-\dd t^2+\dd x^2+\dd y^2+\dd z^2,
\end{equation}
carrega consigo diversos campos de Killing. Considere os seguintes,
\begin{equation}
    \xi^a=\left(\partial_t\right)^a\hspace{1cm}\text{e}\hspace{1cm}\chi^a=a\left[x\left(\partial_t\right)^a+t\left(\partial_x\right)^a\right].
\end{equation}
O primeiro, tem suas órbitas associadas com translação temporal e imediatamente segue que um observador com 4-velocidade dada por \(\xi^a\) tem 4-aceleração nula, logo correspondem a observadores inerciais. Agora, se um observador tem 4-velocidade dada por \(\chi^a\) normalizado, ou seja, \(u^a=\chi^a/B^2\) onde \(B^2=-\chi^a\chi_a=a^2(x^2-t^2)\), sua 4-aceleração é
\begin{subequations}
    \begin{align}
        A^a&=\frac{1}{B^2}\chi^b\nabla_b\chi^a=-\frac{1}{B^2}\chi_b\nabla^a\chi^b=-\frac{1}{2B^2}\nabla^a\left(\chi^b\chi_b\right)\\
        &=\frac{1}{2B^2}\nabla^aB^2=\nabla^a\ln{B}.
    \end{align}
\end{subequations}
Portanto, a sua magnitude é
\begin{equation}
    A=\left(A^aA_a\right)^{\frac{1}{2}}=(x^2-t^2)^{-\frac{1}{2}}=\frac{a}{B},
\end{equation}
ou seja, esses observadores estão uniformemente acelerados. 

Além disso, note que o campo \(\chi^a\) divide o espaço-tempo em 4 regiões delimitadas pelas superfícies em que ele se anula, \(\mathfrak{h}_A\) e \(\mathfrak{h}_B\) (que determinam também um horizonte para os observadores acelerados):
\begin{equation}
    \begin{aligned}
        &\text{Região I}=I^-\left(\mathfrak{h}_A\right)\cap I^+\left(\mathfrak{h}_B\right)\hspace{2cm}\text{Região II}=I^+\left(\mathfrak{h}_A\right)\cap I^-\left(\mathfrak{h}_B\right)\\
        &\text{Região III}=J^+(S)\hspace{3.75cm}\text{Região IV}=J^-(S)
    \end{aligned}
\end{equation}

\begin{figure}[ht]
    \centering
    \begin{tikzpicture}

    % Draw the axes
    \draw[-] (-4,0) -- (4,0);
    \draw (0,0.5) node[above]{S};

    % Draw the lines for the Rindler wedges (extended)
    \draw[thick] (0,0) -- (3,3) node[right]{\(\mathfrak{h}_A\)};
    \draw[thick] (0,0) -- (3,-3);
    \draw[thick] (0,0) -- (-3,3) node[left]{\(\mathfrak{h}_B\)};
    \draw[thick] (0,0) -- (-3,-3);
    
    % Label the regions
    \node at (3,1) {Região I};
    \node at (-3,1) {Região II};
    \node at (0,-2) {Região IV};
    \node at (0,2) {Região III};

\end{tikzpicture}
    \caption{Quadrantes de Rindler}
    \label{fig:wedges}
\end{figure}

Agora, é conveniente definir os tempos (inerciais) retardado, \(u=t-x\), e avançado, \(v=t+x\). Estudando o tempos de Killing induzidos por \(\chi^a\) em \(\mathfrak{h}_A\), \(V\;(\chi^a\nabla_aV=1)\), e em \(\mathfrak{h}_B\), \(U\;(\chi^a\nabla_aU=1)\), temos a relação
\begin{equation}
    V=\frac{1}{a}\ln{\lvert v\rvert}\hspace{1cm}\text{e}\hspace{1cm}U=\frac{1}{a}\ln{\lvert u\rvert}.
\end{equation}

Além disso, tomando as transformadas de Fourier de modos de frequência positiva com relação ao tempo inercial retardado (avançado) em \(\mathfrak{h}_A\;(\mathfrak{h}_B)\), concluímos que estes também são de frequência positiva com respeito ao tempo inercial \(t\).

\subsection*{Banho térmico de Unruh}
Agora, aplicamos a quantização baseada na simetria do campo de Killing tanto para \(\xi^a\) quanto para \(\chi^a\). Para o primeiro, a construção é a usual e denotamos seu espaço de Hilbert por \(\h_1\) e seu estado de vácuo por \(\ket{0_M}\). 

No caso de \(\chi^a\), note que ele é tipo-tempo apenas nas regiões I e II, porém ambas as regiões são espaços-tempos globalmente hiperbólicos por si só. Portanto, aplicamos a quantização em cada uma delas e o espaço de Hilbert de uma partícula será o produto tensorial dos dois, isto é, \(\h_2=\h_I\otimes\h_{II}\). Denotaremos o vácuo dessa construção por \(\ket{0}_R\).

Seja \(\{\psi_i^I\}\) e \(\{\psi_i^{II}\}\) bases das soluções nas regiões I e II respectivamente, em que o índice \(i\) rotula a frequência \(\omega_i\) dos modos. Analisando modos de frequência positiva no horizonte e notando que a operação de reflexão, \((t,x,y,z)\to(-t,-x,y,z)\) mapeia \(\h_I\) em \(\overline{\h}_{II}\), podemos escrever elementos de \(\h_1\) na base de \(\h_2\),
\begin{subequations}
    \begin{align}
        \Psi_i&=\psi_i^I+e^{-\frac{\pi\omega_i}{a}}\left(\psi_i^{II}\right)^*\\
        \Psi'_i&=\psi_i^{II}+e^{-\frac{\pi\omega_i}{a}}\left(\psi_i^{I}\right)^*.
    \end{align}
\end{subequations}

Isto é justamente o que é necessário para computar a matriz-\(S\)\footnote{Também chamada de Transformação de Bogoliubov.} que relaciona as diferentes construções. Assim, os observadores que seguem as órbitas de \(\chi^a\) interpretam o vácuo de Minkowski como
\begin{equation}
    \hat{S}\ket{0}_M=\prod_j\left(\sum_{n=0}^{\infty}e^{-\frac{\pi n\omega_j}{a}}\ket{n_j}_I\otimes\ket{n_j}_{II}\right),
\end{equation}
em que \(\ket{n_j}_I\) representa um estado com \(n\) partículas no modo \(\psi_j^I\) da região I e o mesmo segue para a região II. Como observadores acelerados estão restritos a região I, tomamos o traço parcial da matriz de densidade correspondente ao estado acima e obtemos
\begin{equation}
    \rho^I=\prod_j\left(\sum_{n=0}^{\infty}e^{-\frac{2n\omega_j}{\pi}\ket{n_j}_I\bra{n_j}_I}\right).
\end{equation}

Esse resultado é compatível com a afirmação de que observadores acelerados interpretam o vácuo de Minkowski como um banho térmico de partículas a temperatura
\begin{equation}
    T=\frac{a}{2\pi}\cong\frac{a}{10^{19}\unit{\meter/\second}}\unit{\kelvin}.
\end{equation}

\section{Radiação por uma carga acelerada}\label{sec:eletro}
\subsection*{Solução das equações de Maxwell}
Considere o espaço-tempo de Minkowski \((\mathbb{R}^4,\eta_{ab})\). Para encontrar a radiação emitida por uma carga acelerada, desejamos obter a solução das equações de Maxwell com fontes
\begin{subequations}
    \begin{align}
        &\rho(t,\x)=q\delta(\x-\xq(t))\\
        &\mathbf{J}(t,\x)=\diff{\xq}{t}(t)q\delta(\x-\xq(t)),
    \end{align}
\end{subequations}
em que \(q\) é a carga da partícula e \(\xq(t)\) é a linha de mundo dela.

Escolhendo o Gauge de Lorenz, escrevemos os potenciais a partir da solução retardada,
\begin{subequations}
    \begin{align}
        &\phi(t,\x)=q\frac{\mu_0}{4\pi}\int\dd^3x'\frac{\delta(\x'-\xq(\tr))}{\lvert\x-\x'\rvert}\\
        &\mathbf{A}(t,\x)=q\frac{\mu_0}{4\pi}\int\dd^3x'\diff{\xq}{t}(\tr)\frac{\delta(\x'-\xq(\tr))}{\lvert\x-\x'\rvert},
    \end{align}
\end{subequations}
em que
\begin{equation}
    \tr=t-\lvert\x-\x'\rvert.
\end{equation}

Para avaliar as integrais, note que, para um função vetorial \(\mathbf{f}(\x)\) e uma função escalar \(g(\x)\), vale 
\begin{equation}
    \int\dd^3xg(\x)\delta(\mathbf{f}(\x))=\frac{g}{\lvert\mathcal{J}\rvert}\bigg\vert_{\mathbf{f}(\x)=0},
\end{equation}
onde \(\mathcal{J}\) é o determinante da matriz jacobiana
\begin{equation}
    \mathcal{J}^i_j=\diffp{f^i}{x^j}.
\end{equation}
Este resultado segue de uma transformação do tipo \(\mathbf{y}=\mathbf{f}(\x)\). Para o nosso caso de interesse,
\begin{equation}
    \mathcal{J}=1-\left(\frac{\x-\x'}{\lvert\x-\x'\rvert}\right)\cdot\diff{\xq}{t}(\tr).
\end{equation}
Agora, resolvendo as integrais dos potenciais, obtemos
\begin{subequations}
    \begin{align}
        &\phi(t,\x)=\frac{\mu_0}{4\pi}\frac{1}{\alpha}\frac{q}{\lvert\x-\xq(\tr)\rvert}\\
        &\mathbf{A}(t,\x)=\diff{\xq}{t}(\tr)\phi(t,\x),
    \end{align}
\end{subequations}
com
\begin{equation}
    \alpha=1-\hat{\mathbf{n}}\cdot\diff{\xq}{t}(\tr)\hspace{1cm}\text{e}\hspace{1cm}\hat{\mathbf{n}}=\frac{\x-\xq(\tr)}{\lvert\x-\xq(\tr)\rvert}.
\end{equation}
Esses são os conhecidos potenciais de Liènard-Wiechert para cargas pontuais. Calculando os campos como prescrito para o Gauge de Lorenz, notando que \(\tr\) é função de \(t\) e \(\x\), temos
\begin{subequations}
    \begin{align}
        &\mathbf{E}(t,\x)=q\frac{\mu_0}{4\pi\alpha^3}\left[\frac{\left(1-\big\lvert\diff{\xq}{t}\big\rvert^2\right)\left(\hat{\mathbf{n}}-\diff{\xq}{t}\right)}{\lvert\x-\xq\rvert^2}+\frac{\hat{\n}\times\left[\left(\hat{\n}-\diff{\xq}{t}\right)\times\diff[2]{\xq}{t}\right]}{\lvert\x-\xq\rvert}\right]\Bigg\rvert_{t=\tr}\\
        &\mathbf{B}(t,\x)=\hat{\n}\times\mathbf{E}(t,\x).
    \end{align}
    \label{eq:campos}
\end{subequations}

\subsection*{Potência irradiada}
A potência irradiada para o infinito por unidade de ângulo sólido é definida como
\begin{equation}
    \diff{P}{\Omega}=\lim_{\lvert\x\rvert\to\infty}\lvert\x\rvert^2\mathbf{S}\cdot\er,
\end{equation}
em que \(\mathbf{S}\) é o vetor de Poynting e \(\er=\x/\lvert\x\rvert\). Olhando para a \cref{eq:campos}, apenas o segundo termo contribui para a radiação emitida devido ao expoente no denominador, já que o primeiro caí rápido o suficiente no infinito. Assim, temos
\begin{equation}
    \diff{P}{\Omega}=\frac{q^2\mu_0}{16\pi^2\alpha^6}\left\lvert\er\times\left[\left(\er-\diff{\xq}{t}\right)\times\diff[2]{\xq}{t}\right]\right\rvert.
\end{equation}

No entanto, como a expressão é naturalmente avaliada em esferas com tempo retardado constante, é conveniente definir a potência irradiada por unidade de tempo retardado, isto é,
\begin{equation}
    \diff{P'}{\Omega}\equiv\diff{P}{\Omega}\diff{t}{\tr}=\frac{q^2\mu_0}{16\pi^2\alpha^5}\left\lvert\er\times\left[\left(\er-\diff{\xq}{t}\right)\times\diff[2]{\xq}{t}\right]\right\rvert.
\end{equation}

Considere o caso em que a partícula está instantaneamente em repouso, \(\diff{\xq}{t}=0\) e \(\alpha=1\), temos
\begin{equation}
    \diff{P}{t}=\frac{q^2\mu_0}{16\pi^2}\left[\left\lvert\diff[2]{\xq}{t}\right\rvert^2-\left(\er\cdot\diff[2]{\xq}{t}\right)^2\right].
\end{equation}
Integrando sobre as variáveis angulares para obter a radiação total emitida num instante de tempo,
\begin{equation}
    P_0=\int\diff{P}{\Omega}\dd\Omega=\frac{q^2\mu_0}{6\pi}\left\lvert\diff[2]{\xq}{t}\right\rvert^2=\frac{q^2\mu_0}{6\pi}a^\mu a_\mu,
\end{equation}
em que \(a^\mu\) é a 4-aceleração da partícula. Essa é a famosa fórmula de Larmor para radiação emitida por uma carga acelerada. Perceba que essa expressão é idêntica para a radiação emitida por tempo retardado no referencial de repouso da carga. Como tanto o 4-momento (energia) quanto o tempo recebem um fator \(\gamma\) quando transformado entre referenciais, a potência por tempo retardado é dada por
\begin{equation}
    P'=\frac{q^2\mu_0}{6\pi}a^\mu a_\mu,
\end{equation}
independente do estado de movimento da partícula.

\section{Conexão entre radiação e o efeito Unruh}
Um tratamento cuidadoso da radiação emitida por uma carga acelerada mostra que esse conceito não é covariante\cite{boulware}, isto é, um observador acelerado junto com a carga não detecta radiação emitida por ela. No entanto, no contexto de TQCEC, entendemos que esse observador explica a emissão desses fótons (inerciais) como absorção e emissão de fótons de Rindler (do banho térmico de Unruh), especialmente dos fótons de energia nula. Desejamos mostrar essa equivalência a seguir para um carga escalar\footnote{Algumas expressões diferem das encontradas na \cref{sec:eletro} devido a característica escalar do campo, portanto comparamos com o a Ref\cite{ren}.}.

\subsection*{Análise clássica}
Considere uma fonte escalar \(j\) (com carga \(q\)) uniformemente acelerada que interage com um campo escalar clássico \(\phi\) por um tempo próprio \(\Ttot=2T\) no espaço-tempo de Minkowski \((\mathbb{R}^4,\eta_{ab})\). A fonte escalar é dada por
\begin{equation}
    j(\x)=
    \begin{cases}
        q\delta(\xi)\delta^2(\xperp),\;-T<\tau<T\\
        0,\;\;\lvert\tau\rvert>T
    \end{cases}
\end{equation}
em que \(\tau\) é o tempo próprio da partícula. Nas coordenadas de Rindler, \((\tau,\xi,\xperp)\), com \(\xperp=(x,y)\in\mathbb{R}^2\), que cobre o quadrante de Rindler da direita (região \(z>\lvert t\rvert\)), o elemento de linha é
\begin{equation}
    \dd s^2=e^{2a\xi}(-\dd\tau^2+\dd\xi^2)+\dd x^2+\dd y^2.
\end{equation}
Além disso, nessas coordenadas a linha de mundo da carga uniformemente acelerada é dada por \(\xi=x=y=0\).

O acoplamento da fonte com o campo escalar leva a equação de Klein-Gordon não homogênea,
\begin{equation}
    \nabla^a\nabla_a\phi=j,
\end{equation}
que tem solução retardada denotada por \(Rj\). Tomando uma superfície de Cauchy \(\Sigma\subset\mathbb{R}^4-J^-(\supp{j})\), então nessa superfície, temos
\begin{equation}
    Rj=-(Aj-Ej)=-Ej,
\end{equation}
visto que a solução avançada é nula fora do passado do suporte de \(j\).

Para analisar a radiação emitida no futuro assintótico, decompomos a solução retardada nos modos de Unruh \(\{\wu{\omega},\wv{\omega}\}\), em que \(\omega\in\mathbb{R}^+\) e \(\mathbf{k}_\perp\in\mathbb{R}^2\) são a frequência e momento dos modos de Rindler, associado com os observadores acelerados. No entanto, os modos de Unruh, assim como os usados na dedução do efeito Unruh, tem frequência positiva com respeito ao observadores inerciais. A expressão desses modos é
\begin{subequations}
    \begin{align}
        &\wu{\omega}=\frac{v^R_{\omega\mathbf{k}_\perp}+e^{-\pi\omega/a}v^{L*}_{\omega-\mathbf{k}_\perp}}{\sqrt{1-e^{-2\pi\omega/a}}}
        &\wv{\omega}=\frac{v^L_{\omega\mathbf{k}_\perp}+e^{-\pi\omega/a}v^{R*}_{\omega-\mathbf{k}_\perp}}{\sqrt{1-e^{-2\pi\omega/a}}},
    \end{align}
\end{subequations}
em que \(v^R_{\omega\mathbf{k}_\perp}\) e \(v^L_{\omega\mathbf{k}_\perp}\) representam os modos de Rindler nos quadrantes da direita e esquerda respectivamente e tem a forma
\begin{equation}
    v^R_{\omega\mathbf{k}_\perp}=e^{-i\omega\tau}F_{\omega\mathbf{k}_\perp}(\xi,\xperp),
\end{equation}
com
\begin{equation}
    F_{\omega\mathbf{k}_\perp}(\xi,\xperp)=\left[\frac{\sinh{\pi\omega/a}}{4\pi^4a}\right]^{1/2}K_{i\omega/a}\left(\frac{k_\perp}{a}e^{a\xi}\right)e^{i\mathbf{k}_\perp\cdot\xperp},
\end{equation}
em que \(K_\nu(x)\) é a função de Bessel modificada de ordem \(\nu\).

Lembrando que \(\nabla^a\nabla_aRj=0\) fora de \(\supp{j}\) e fazendo a expansão explicitamente em \(\Sigma\) em termo dos modos de Unruh, temos
\begin{equation}
    Rj=-\sum_{\sigma=1}^2\int_0^{\infty}\dd\omega\int\dd^2\mathbf{k}_\perp\innerkg{\wsig{\omega}}{Ej}\wsig{\omega}+\text{c.c.}
\end{equation}

Invocando a \cref{eq:int-inner} para calcular os coeficientes da expansão e tomando o limite em que a carga acelera por um tempo próprio infinito \((T\to\infty)\), obtemos
\begin{equation}
    \innerkg{\wsig{\omega}}{Ej}=\frac{iqK_0(k_\perp/a)}{\sqrt{2\pi^2a}}\delta(\omega),\;\sigma=1,2.
\end{equation}

Agora, usando a propriedade que \(\wu{-\omega}=\wv{\omega}\) para calcular a solução retardada,
\begin{equation}
    Rj=-\frac{iq}{\sqrt{2\pi^2a}}\int\dd^2\mathbf{k}_\perp K_0(k_\perp/a)\wv{0}+\text{c.c.}
\end{equation}

Analisando a expressão acima é evidente que apenas os modos de Unruh de energia nula contribuem para a construção da solução retardada como vista por observadores inerciais no futuro assintótico. Agora, como não é possível medir a radiação na cunha da direita\cite{boulware}, performando o cálculo explícito no quadrante futuro \((t>\lvert z\rvert)\) usando coordenadas polares, obtemos
\begin{equation}
    Rj=\frac{-q}{4\pi\rho_0(x)},
\end{equation}
em que \(\rho(x)\equiv a/2\sqrt{(-x^\mu x_\mu+a^{-2})^2+4(t^2-z^2)/a^2}\). Esse é justamente o resultado usual da eletrodinâmica (escalar)\cite{ren}.

Para finalizar a analise clássica, podemos definir um número clássico de particulas irradiadas (como visto por observadores inerciais) como
\begin{equation}
    N_M\equiv\innerkg{KRj}{KRj},
\end{equation}
e usando a ortonormalidade dos modos de Unruh,
\begin{equation}
    \frac{N_M}{\Ttot}=\frac{q^2a}{4\pi^2},
\end{equation}
onde também foi usado \(\Ttot=2\pi\delta(\omega)\vert_{\omega=0}\) e 
\begin{equation}
    \int_0^{\infty}\dd x\;x\lvert K_0(x)\rvert=\frac{1}{2}.
\end{equation}

\subsection*{Análise quântica}
Agora, a fonte está acoplada com um campo quântico
\begin{equation}
    \nabla^a\nabla_a\hat{\phi}=j,
\end{equation}
cuja solução pode ser escrita em termos das soluções retardada e avançada, ou seja,
\begin{subequations}
    \begin{align}
        \hat{\phi}(t,\x)&=Rj(t,\x)\mathbb{1}+\hat{\phi}_{\text{in}}(t,\x)\\
        &=Aj(t,\x)\mathbb{1}+\hat{\phi}_{\text{out}}(t,\x),
    \end{align}
\end{subequations}
em que \(\hat{\phi}_{\text{in}},\hat{\phi}_{\text{out}}\) são soluções homogêneas da equação de Klein-Gordon. Assim, podemos expandir as soluções homogêneas numa base de soluções de frequência positiva e nos operadores de aniquilação e criação, por exemplo,
\begin{equation}
    \hat{\phi}_{\text{out}}(t,\x)=\sum_j\left[u_j(t,\x)\hat{a}_{\text{out}}(u_j^*)+u_j^*(t,\x)\hat{a}^{\dagger}(u_j)\right]
\end{equation}
No entanto, cada uma delas tem seu estado de vácuo associado, isto é, \(\ket{0_\text{in}}\) e \(\ket{0_{\text{out}}}\) (e consequentemente seus operadores de aniquilação e criação), o primeiro associado com observadores no passado assintótico e o segundo com observadores no futuro assintótico.

Ambos os estados podem ser conectados pela matriz-\(S\) dada por
\begin{equation}
    \hat{S}=\exp{\left[-i\int\dd x^4\sqrt{-g}\hat{\phi}_{\text{out}}(x)j(x)\right]}.
\end{equation}

Portanto, preparando o campo no estado \(\ket{0_{\text{in}}}\), temos
\begin{equation}
    \ket{0_{\text{in}}}=\hat{S}\ket{0_{\text{out}}}.
\end{equation}
Perceba que podemos escrever a matriz-\(S\) como função do operador campo como definido na \cref{eq:campo},
\begin{subequations}
    \begin{align}
        \hat{S}&=\exp[-i\hat{\phi}(j)]\\
        &=\exp{\left[\hat{a}\left(KEj^*\right)-\hat{a}^{\dagger}\left(KEj\right)\right]}.
    \end{align}
\end{subequations}

Usando a fórmula de Zassenhaus, 
\begin{equation}
    e^{A+B}=e^{A}e^{B}e^{-\frac{1}{2}[A,B]},
\end{equation}
quando \([A,B]\) é proporcional a identidade, com \(A=\hat{a}^{\dagger}\left(KEj\right)\) e \(B=\hat{a}\left(KEj^*\right)\), chegamos na relação entre os estados de vácuo in e out,
\begin{equation}
    \ket{0_{\text{in}}}=e^{-\frac{\lVert KEj\rVert^2}{2}}e^{-\hat{a}^{\dagger}\left(KEj\right)}\ket{0_{\text{out}}}.
    \label{eq:coerente}
\end{equation}
Note que este é um estado coerente, ou seja, é autoestado de \(\hat{a}_{\text{out}}\left(w^{\sigma*}_{\omega\mathbf{k}_\perp}\right)\) com autovalor \(-\innerkg{\wsig{\omega}}{Ej}\).

Com esse resultado e expandindo o campo nos modos de Unruh, 
\begin{equation}
    \hat{\phi}_{\text{out}}=\sum_{\sigma=1}^2\int_{0}^{\infty}\dd\omega\int\dd^2\mathbf{k}_\perp\left[\wsig{\omega}(x)\hat{a}_{\text{out}}\left(w^{\sigma*}_{\omega\mathbf{k}_\perp}\right)+w^{\sigma*}_{\omega\mathbf{k}_\perp}(x)\hat{a}^{\dagger}_{\text{out}}\left(\wsig{\omega}\right)\right],
\end{equation}
é imediato que o valor esperado do campo é
\begin{equation}
    \bra{0_{\text{in}}}\hat{\phi}_{\text{out}}\ket{0_\text{in}}=-\sum_{\sigma=1}^2\int_0^{\infty}\dd\omega\int\mathbf{k}_\perp\innerkg{\wsig{\omega}}{Ej}\wsig{\omega}+\text{c.c.},
\end{equation}
que é \(-Ej\). No entanto, em \(\Sigma\) (futuro assintótico), temos \(Rj=-Ej\), então
\begin{equation}
    \bra{0_{\text{in}}}\hat{\phi}_{\text{out}}\ket{0_\text{in}}=Rj,
\end{equation}
que é a solução clássica encontrada na análise anterior. Não somente o valor esperado do campo é construído com apenas modos de energia nula, mas o estado em si também o é. Isto pode ser verificado expandindo o operador de criação \(\hat{a}^\dagger(KEj)\) em modos de Unruh e aplicar na \cref{eq:coerente} tomando o limite \(T\to\infty\) que resulta em
\begin{equation}
    \ket{0_{\text{in}}}=\exp\left[-\Ttot q^2a/(4\pi^2)\right]\bigotimes_{\mathbf{k}_\perp}\exp\left[\frac{iqK_0(k_\perp/a)}{\sqrt{2\pi^2a}}a^{\dagger}_{\text{out}}\left(\wv{\omega}\right)\right].
\end{equation}

Esse resultado concorda com a afirmação de que cada partícula emitida no referencial inercial corresponde, no referencial acelerado, a emissão ou absorção de partículas de Rindler de energia nula.

Para finalizar a análise, abusamos do fato do estado no futuro assintótico ser coerente para calcular o número esperado de partículas emitidas em cada modo de Unruh,
\begin{equation}
    \bra{0_{\text{in}}}\hat{N}^{\text{out}}_{\omega\mathbb{k}_\perp}\ket{0_\text{in}}=\lvert\innerkg{\wsig{\omega}}{Ej}\rvert^2,
\end{equation}
em que \(\hat{N}^{\text{out}}_{\omega\mathbf{k}_\perp}=\hat{a}^{\dagger}_{\text{out}}(\wsig{\omega})\hat{a}_{\text{out}}(w^{\omega*}_{\omega\mathbf{k}_\perp}\) é o operador número. Integrando sobre todos os modos, obtemos
\begin{equation}
    \frac{\bra{0_{\text{in}}}\hat{N}^{\text{out}}\ket{0_\text{in}}}{\Ttot}=\frac{q^2a}{4\pi^2},
\end{equation}
que concorda com o resultado clássico anterior. Por fim, calculando o valor médio para o tensor de energia-momento normalmente ordenado,
\begin{equation}
    :\hat{T}^{\text{out}}_{ab}:=\hat{T}^{\text{out}}_{ab}-\bra{0_\text{out}}\hat{T}^{\text{out}}_{ab}\ket{0_{\text{out}}},
\end{equation}
chegamos na expressão clássica
\begin{equation}
    \bra{0_{\text{in}}}:\hat{T}^{\text{out}}_{ab}:\ket{0_\text{in}}=\nabla_aRj\nabla_bRj-\frac{1}{2}\eta_{ab}\nabla^cRj\nabla_cRj.
\end{equation}

Particularmente, a fluxo de energia integrado numa esfera no futuro assintótico é
\begin{equation}
    \int\dd S^b \bra{0_{\text{in}}}:\hat{T}^{\text{out}}_{ab}:\ket{0_\text{in}}\left(\partial_t\right)^a=\frac{q^2a^2}{12\pi},
\end{equation}
que é a fórmula de Larmor para uma carga escalar (com respeito aos observadores inerciais)\cite{ren}. Aqui, \(\dd S^b\) é um vetor de um elemento de volume na esfera.

Finalmente, concluímos que os modos de Rindler de energia nula tem papel crucial na conexão entre a emissão de radiação de uma carga acelerada e o efeito Unruh no limite de aceleração por um tempo próprio infinito. Além disso, encontramos uma correspondência entre os valores clássicos e quânticos.

\section{Próximos passos}
No período de vigência que segue, pretendemos estudar um problema original: como um observador (detector) acelerado enxerga um espaço-tempo de Minkowski com uma carga escalar em repouso. 

Além disso, pretendemos estudar outros efeitos paradigmáticos importantes da teoria, como por exemplo, o efeito Hawking. Dependendo do seguimento, também é pretendido estudar a formulação algébrica da teoria quântica de campos em espaços-tempos curvos.

\bibliographystyle{plain}
\bibliography{refs}
\nocite{*}
\end{document}