\documentclass[11pt, a4paper]{article}

%imagens
\usepackage{tikz,graphicx}
\usepackage{caption,subcaption}
\usepackage{float}

%matemáticas
\usepackage{amsmath,amsfonts,amssymb,amsthm}
\usepackage[ISO]{diffcoeff}
\usepackage{braket,esint}

%formatação
\usepackage[T1]{fontenc}
\usepackage{inputenc}
\usepackage[portuguese]{babel}
\usepackage{lmodern}
\usepackage{indentfirst}
\usepackage{orcidlink}

%referências
\usepackage{hyperref}
\usepackage[capitalize, brazilian, nameinlink, noabbrev]{cleveref}
\hypersetup{ 
colorlinks = true,
linkcolor = blue,
filecolor = black,
urlcolor = black
}

\title{Radiação, Aceleração e o Efeito Unruh}
\author{Carlos H. Correr da Silva\orcidlink{0000-0002-3598-4412}}


\begin{document}

\begin{titlepage}
    \begin{center}
        \vspace{5cm}
        \makeatletter
        {\huge\@title\\
        \vspace{0.5cm}
        \LARGE \href{https://bv.fapesp.br/pt/pesquisa/buscador/?q2=(id_pesquisador_exact%3A724334)%20AND%20(bolsa_exact:%22Bolsas%20no%20Brasil%22%20AND%20situacao:%22Em%20andamento%22)}{Processo: 24/01141-1}}\\
        \hrulefill \\
        \vspace{0.5cm}
        {\LARGE\@author
        \vspace{0.5cm}\\
        Orientador: André G. S. Landulfo}\\
        \vspace{2cm}
        {\LARGE Vigência: 01/05/2024 a 30/04/2025\\
        \vspace{0.2cm}
        Período coberto: 01/05/2024 a 10/10/2024}
    \end{center}
    \vspace{6cm}
    Resumo: A conexão entre aceleração e radiação e sua relação com o princípio da equivalência vêm intrigando a comunidade científica a décadas. Mais recentemente, esta questão foi investigada da perspectiva da mecânica quântica no âmbito da teoria quântica de campos em espaços-tempos curvos e foi encontrado o que parece ser uma conexão impressionante entre o  bremsstrahlung (um efeito clássico) e o efeito Unruh (um efeito puramente quântico). Mais do que isso, os chamados fótons de Rindler de {\em energia nula} desempenham um papel surpreendente, porém crucial, em tal conexão. Neste projeto de iniciação científica, pretendemos estudar os aspectos clássicos e quânticos da radiação emitida por uma carga (escalar) acelerada e o papel exato desempenhado pelos fótons de Rindler de energia nula (já no contexto clássico). Tal análise servirá como "cavalo de batalha" para introduzir o estudante à teoria quântica de campos em espaços-tempos curvos e seus mais diversos aspectos conceituais.
\end{titlepage}

\hrule
\tableofcontents
\hrule

\section{Graduação}
Ao longo do período coberto, conclui o meu 5° semestre da graduação, em que cursei as seguintes  matérias: Mecânica Quântica I, Eletromagnetismo I, Termodinâmica, Mecânica I e Física Experimental V. As bases teóricas de quântica estudadas para o projeto complementaram-se e solidificaram-se com o andamento da disciplina cursada. Além disso, a matéria de eletromagnetismo forneceu o arcabouço para o estudo da radiação emitida por uma carga acelerada. 

No fim do período, estou na metade do 6° semestre, cursando as disciplinas: Relatividade Geral, Eletromagnetismo II, Grupos e Tensores, Mecânica Estatística e Física Computacional I. A matéria de relatividade está reforçando os conceitos que estudei antes do início do projeto e complementando o incentivo com diversos exercícios. Enquanto isso, eletromagnetismo está enfatizando a importância dos potenciais eletromagnéticos e as liberdades de escolha que permitem obter equações de onda para as grandezas relevantes, mantendo a física do sistema de estudo.

Ainda por cima, participei dos Journal Clubs do grupo, que permite o contato com diversas tópicos atuais de pesquisa na área. Também apresentei um pôster sobre o Efeito Unruh na VI Escola Jayme Tiomno, uma escola de inverno do IFUSP, na qual sou um dos organizadores.

\section{Teoria quântica de campos em espaços-tempos curvos}

\section{Radiação por uma carga acelerada}

\end{document}